%%%%%%%%%%%%%%%%%%%%%%%%%%%%%%%%%%%%%%%%%
% Short Sectioned Assignment
% LaTeX Template
% Version 1.0 (5/5/12)
%
% This template has been downloaded from:
% http://www.LaTeXTemplates.com
%
% Original author:
% Frits Wenneker (http://www.howtotex.com)
%
% License:
% CC BY-NC-SA 3.0 (http://creativecommons.org/licenses/by-nc-sa/3.0/)
%
% Author:
% Guangcheng Wei
%%%%%%%%%%%%%%%%%%%%%%%%%%%%%%%%%%%%%%%%%

%ref
% MI template http://www.cis.syr.edu/courses/cis428/DMI.pdf

% x mark from https://tex.stackexchange.com/questions/42619/x-mark-to-match-checkmark

% homework questions http://www.cis.syr.edu/courses/cis428/hw01.pdf
% Latex reference https://en.wikibooks.org/wiki/LaTeX
% slide http://www.cis.syr.edu/courses/cis428/slides/01mono.pdf

%----------------------------------------------------------------------------------------
%   PACKAGES AND OTHER DOCUMENT CONFIGURATIONS
%----------------------------------------------------------------------------------------

\documentclass[paper=a4, fontsize=11pt]{scrartcl} % A4 paper and 11pt font size

\usepackage[T1]{fontenc} % Use 8-bit encoding that has 256 glyphs
\usepackage{fourier} % Use the Adobe Utopia font for the document - comment this line to return to the LaTeX default
\usepackage[english]{babel} % English language/hyphenation

\usepackage{amssymb}% http://ctan.org/pkg/amssymb
\usepackage{pifont}% http://ctan.org/pkg/pifont
\newcommand{\xmark}{\ding{55}}%

\usepackage{amsmath,amsfonts,amsthm} % Math packages
\usepackage{listings} % Code package
\usepackage{lipsum} % Used for inserting dummy 'Lorem ipsum' text into the template

\usepackage{sectsty} % Allows customizing section commands
\allsectionsfont{\normalfont\scshape} % Make all sections centered, the default font and small caps

\usepackage{fancyhdr} % Custom headers and footers
\pagestyle{fancyplain} % Makes all pages in the document conform to the custom headers and footers
\fancyhead{} % No page header - if you want one, create it in the same way as the footers below
\fancyfoot[L]{} % Empty left footer
\fancyfoot[C]{} % Empty center footer
\fancyfoot[R]{\thepage} % Page numbering for right footer
\renewcommand{\headrulewidth}{0pt} % Remove header underlines
\renewcommand{\footrulewidth}{0pt} % Remove footer underlines
\setlength{\headheight}{13.6pt} % Customize the height of the header

\numberwithin{equation}{section} % Number equations within sections (i.e. 1.1, 1.2, 2.1, 2.2 instead of 1, 2, 3, 4)
\numberwithin{figure}{section} % Number figures within sections (i.e. 1.1, 1.2, 2.1, 2.2 instead of 1, 2, 3, 4)
\numberwithin{table}{section} % Number tables within sections (i.e. 1.1, 1.2, 2.1, 2.2 instead of 1, 2, 3, 4)

\setlength\parindent{0pt} % Removes all indentation from paragraphs - comment this line for an assignment with lots of text

%----------------------------------------------------------------------------------------
%   TITLE SECTION
%----------------------------------------------------------------------------------------

\newcommand{\problem}[1]{\subsection *{Problem #1}}
\newcommand{\claim}{\textbf{Claim}: }
\newcommand{\basis}{\textbf{Basis:}: }
\newcommand{\inductiveStep}{\textbf{Inductive step:}: }
\newcommand{\pnl}{$ $\newline\\}
\newcommand{\N}{\mathbb{N}}
\newcommand{\Z}{\mathbb{Z}}
\newcommand{\horrule}[1]{\rule{\linewidth}{#1}} % Create horizontal rule command with 1 argument of height

\title{ 
\horrule{0.5pt} \\[0.4cm] % Thin top horizontal rule
\huge Homework 1: A Math Warm-Up \\ % The assignment title
\horrule{2pt} \\[0.5cm] % Thick bottom horizontal rule
}

\author{Guangcheng Wei (799233472)} % Your name
\date{\normalsize\today} % Today's date or a custom date

\begin{document}

\maketitle % Print the title

1.  Assignment: Homework 1\\

2a. Name \& email address (First author): Guangcheng Wei<gwei100@syr.edu> \\
2b. Name \& email address (Second author, if any):\\
\\
3a. Did you consult with anyone on parts of this assignment? (Yes/No): No\\
3b. If so, give the details of these consultations (e.g., who, what,    where)?\\
\\
4a. Did you consulted an outside source on parts of this assignment     (e.g., an Internet site)?  (Yes/No): Yes
4b. If so, give the details of these consultations (e.g., who, what,     where)?
\\
Assignment Latex Template(Frits Wenneker) http://www.latextemplates.com/template/short-sectioned-assignment\\
Latex Reference https://en.wikibooks.org/wiki/LaTeX\\
xmark \xmark in Latex (Werner) from https://tex.stackexchange.com/questions/42619/x-mark-to-match-checkmark\\
\\
Homework questions(Jim Royer) http://www.cis.syr.edu/courses/cis428/hw01.pdf\\
Slide(Jim Royer) http://www.cis.syr.edu/courses/cis428/slides/01mono.pdf\\
DMI template(Jim Royer) http://www.cis.syr.edu/courses/cis428/DMI.pdf\\
\\
5.  The authors attest that the above is correct, (Yes/No): Yes\\

\pagebreak
\section {Mathematical Induction Arguments}

\problem 1 Prove by induction on $n$ that, for each $n \geq 1,$
\[1+3+5+\cdots+(2n-1) = n^2.\]

\begin{proof}
\pnl
By mathematical induction\\
Let P(n) be the statement: $1+3+5+\cdots+(2n-1) = n^2$\\
\basis P(1) is true, because $1 = 2*1 - 1 = 1^2$\\
\inductiveStep Suppose $k \geq 1$. For the inductive hypothesis, suppose P(k) is true: \\
$$1+3+5+\cdots+(2k-1) = k^2$$
We need to show that P(k+1) is true:
$$1+3+5+\cdots+(2(k+1)-1) = (k+1)^2$$
$\because \ (k+1)^2 = k^2 + 2k + 1 \ and \ P(k)$\\\\
$\therefore \ (k+1)^2 =1+3+5+\cdots+(2k-1)+2k+1$\\\\
To prove P(k+1), we need to show that:\\\\
$(2(k+1)-1) = 2k+1$\\\\
$\because \ (2(k+1)-1) = 2k + 2 - 1 = 2k + 1$\\\\
$\therefore \ (k+1)^2 =1+3+5+\cdots+2(k-1)+2(k+1)$\\\\
Thus, for all $k \geq 1$ , the conditional P(k) $\implies$ P(k + 1) is true.\\\\
By the basis, inductive step, and the principle of mathematical induction, the claim is true.
\end{proof}

\problem 2 Prove by induction on $n$ that, for each $n \geq 1,$
\[1\cdot2+2\cdot3+3\cdot4+\cdots+n\cdot(n+1)=\frac{n\cdot(n+1)(n+2)}{3}\]

\begin{proof}
\pnl
By mathematical induction\\
Let P(n) be the statement: $1\cdot2+2\cdot3+3\cdot4+\cdots+n\cdot(n+1)=\frac{n\cdot(n+1)(n+2)}{3}$\\
\basis P(1) is true, because $1*2 = \frac{1*2*3}{3}$\\
\inductiveStep Suppose $k \geq 1$. For the inductive hypothesis, suppose P(k) is true: \\
$$1\cdot2+2\cdot3+3\cdot4+\cdots+k\cdot(k+1)=\frac{k\cdot(k+1)(k+2)}{3}$$
We need to show that P(k+1) is true:
$$1\cdot2+2\cdot3+3\cdot4+\cdots+(k+1)\cdot(k+2)=\frac{(k+1)\cdot(k+2)(k+3)}{3}$$
$\because \ 1\cdot2+2\cdot3+3\cdot4+\cdots+(k+1)\cdot(k+2) \\= 1\cdot2+2\cdot3+3\cdot4+\cdots+k\cdot(k+1)+(k+1)\cdot(k+2) \ and \ P(k)$\\\\
$\therefore \ 1\cdot2+2\cdot3+3\cdot4+\cdots+(k+1)\cdot(k+2) = \frac{k\cdot(k+1)(k+2)}{3} + (k+1)\cdot(k+2)$\\\\
To prove P(k+1), we need to show that:\\\\
$\frac{k\cdot(k+1)(k+2)}{3} + (k+1)\cdot(k+2) = \frac{(k+1)\cdot(k+2)(k+3)}{3}$\\\\
$\because \ \frac{k\cdot(k+1)(k+2)}{3} + (k+1)\cdot(k+2) = \frac{k\cdot(k+1)(k+2)}{3} + \frac{3\cdot(k+1)\cdot(k+2)}{3} = \frac{(k+3)\cdot(k+1)\cdot(k+2)}{3} \\\\
=\frac{(k+1)\cdot(k+2)(k+3)}{3}$\\\\
$\therefore \ 1\cdot2+2\cdot3+3\cdot4+\cdots+(k+1)\cdot(k+2)=\frac{(k+1)\cdot(k+2)(k+3)}{3}$\\\\
Thus, for all $k \geq 1$ , the conditional P(k) $\implies$ P(k + 1) is true.\\
By the basis, inductive step, and the principle of mathematical induction, the claim is true.
\end{proof}

\problem 3
Prove by induction on $n$ that, for each $n \geq 1, n \leq 2^n.$
\begin{proof}
\pnl
By mathematical induction\\
Let P(n) be the statement: $n \leq 2^n$\\
\basis P(1) is true, because $1 < 2^1$\\
\inductiveStep Suppose $k \geq 1$. For the inductive hypothesis, suppose P(k) is true: \\
$$k \leq 2^k$$
We need to show that P(k+1) is true:
$$k+1 \leq 2^{k+1}$$
$\because \ k+1 \ and \ P(k)$\\\\
$\therefore \ k+1 \leq 2^k + 1$
To prove P(k+1), we need to show that:\\\\
$2^k + 1 \leq 2^{k+1} \ \ (k \geq 1)$\\\\
$\because \  2^{k+1} = 2^k \cdot 2 = 2^k + 2^k \ and \ k \geq 1$\\\\
$\therefore \ 2 \leq 2^k $\\\\
$\therefore \ 2^k + 2 \leq 2^{k+1} $\\\\
$\because \ 2^k + 1 < 2^k + 2 \ \ (k \geq 1)$\\\\
$\therefore \ 2^k + 1 \leq 2^{k+1} \ \ (k \geq 1)$\\\\

Thus, for all $k \geq 1$ , the conditional P(k) $\implies$ P(k + 1) is true.
By the basis, inductive step, and the principle of mathematical induction, the claim is true.
\end{proof}
\section {Divisibility}

The slides presented the following utility lemma about divisibility.
\newline
\newline
\textbf{Lemma}
\renewcommand{\labelenumi}{(\alph{enumi})}
\begin{enumerate}
  \item $d|a$ and $d|b \implies d|b$.
  \item $d|a$ iff $d|(-a)$.
  \item $d|a$ iff $-d|a$.
  \item $\pm 1|a$ for any $a \in \Z $.
  \item $\pm d|0$ for any $d \in \Z^+ $.
  \item $a \neq 0$ \& $d|a \implies |d| \leq |a|$.
  \item $a \neq 0$ \& $d|a$ \& $a \neq \pm d \implies |d| < |a|$.
  \item $d|\pm 1 \implies d = \pm 1$.
  \item $a|b$ \& $b|a \implies a = \pm b$.
  \item $d|a$ \& $d|b \implies (\forall x,y)[d|(ax+by)]$.
\end{enumerate}

\problem 4
Prove part (d) of the lemma. You may use parts (a), (b), and (c).
\begin{proof}
\pnl
$\because \ a = a \cdot 1$ and $a \in \Z.$\\\\
$\therefore \ 1|a$  by definition.\\\\
$\therefore \ -1|a$  by Lemma (c)\\\\
$\therefore \ \pm 1|a$\\\\
\end{proof}

\problem 5
Prove part (h) of the lemma. You may use parts (a) through (g).
\begin{proof}
\pnl
$\because \ d|\pm 1$ and $1 \neq 0$\\\\
$\therefore \ |d| \leq |1|$ by Lemma (f).\\\\
$\because \ \pm 1 | d$ by Lemma (d).\\\\
$\therefore \ |1|\leq|d|$ by Lemma (f).\\\\
$\because \ |1|\leq|d|$ and $|d| \leq |1|$\\\\
$\therefore \ |d|=|1|$\\\\
$\therefore \ d=\pm 1$\\\\

\end{proof}

\problem 6
Prove part (j) of the lemma. You may use parts (a) through (i).
\begin{proof}
\pnl
$\because \ d|a \& d|b$\\\\
$\therefore \ a = p \cdot d$ and $b = q \cdot d, (p, q \in \Z)$\\\\
$\because  (\forall x, y) ax+by = pxd+qyd = (px+qy) d $ and $(px+qy) \in \Z$\\\\
$\therefore (\forall x, y) [d|(ax+by)]$ by defination\\\\
\end{proof}

\section {Divisibility}
\problem 7
Suppose that $a|n$, $b|n$, and $gcd(a, b) = 1$ where $a, b,$ and $n \in \Z^+$.
Prove that $(a \cdot b)|n$.
\begin{proof}
\pnl
To prove $(a \cdot b)|n$, by defination, we need to show that:\\
$$(\exists k \in \Z) n = (a \cdot b) k$$
$\because a, b \in \Z^+$\\\\
$\therefore(\exists k \in \Z) n = (a \cdot b) k \equiv \ (\exists k \in \Z) k = \frac{n}{(a \cdot b)} $\\\\
$\because \ gcd(a, b) = 1$\\\\
$\therefore \ \exists (x,y \in \Z) a\cdot x+b\cdot y = 1 $\\\\
$\therefore \ \exists (x,y \in \Z) k = \frac{n \cdot 1}{(a \cdot b)} = \frac{n(ax+by)}{(a \cdot b)} $\\\\
$\because \ a|n, b|n$\\\\
$\therefore \ n = p\cdot a, n = q\cdot b \ (p,q \in \Z)$\\\\
$\therefore \ k = \frac{n(ax+by)}{(a \cdot b)} = \frac{nax+nby)}{(a \cdot b)} = \frac{nax}{(a \cdot b)} + \frac{nby}{(a \cdot b)} = \frac{nx}{b} + \frac{ny}{a} = qx + py$\\\\
$\therefore \ k = qx+py$\\\\
$\because x,y \in \Z$ and $p,q \in \Z$\\\\
$\therefore \ k \in \Z$\\\\
$\therefore \ (\exists k \in \Z) k = \frac{n}{(a \cdot b)}$\\\\
$\therefore \ (\exists k \in \Z) n = (a \cdot b) k$\\\\
$\therefore \ (a \cdot b)|n$ by defination.\\\\

\end{proof}

\section {Modular Arithmetic}
\problem 8
\renewcommand{\labelenumi}{(\alph{enumi})}
\begin{enumerate}
  \item Write a little program to print out the multiplication table of the numbers {1,...,14} mod 15.
  \item Based on this table, make a table of the numbers 1, . . . , 14 and, for each number, either give its multiplicative inverse (mod 15), if it has one, or else write \xmark \ if it fails to have an inverse.
\end{enumerate}
(a) The program written in javascript shown below.
\lstinputlisting[language]{mulmod15.js}

(a) The table printed is:
\begin{lstlisting}
    1   2   3   4   5   6   7   8   9   10  11  12  13  14
1   1   2   3   4   5   6   7   8   9   10  11  12  13  14
2   2   4   6   8   10  12  14  1   3   5   7   9   11  13
3   3   6   9   12  0   3   6   9   12  0   3   6   9   12
4   4   8   12  1   5   9   13  2   6   10  14  3   7   11
5   5   10  0   5   10  0   5   10  0   5   10  0   5   10
6   6   12  3   9   0   6   12  3   9   0   6   12  3   9
7   7   14  6   13  5   12  4   11  3   10  2   9   1   8
8   8   1   9   2   10  3   11  4   12  5   13  6   14  7
9   9   3   12  6   0   9   3   12  6   0   9   3   12  6
10  10  5   0   10  5   0   10  5   0   10  5   0   10  5
11  11  7   3   14  10  6   2   13  9   5   1   12  8   4
12  12  9   6   3   0   12  9   6   3   0   12  9   6   3
13  13  11  9   7   5   3   1   14  12  10  8   6   4   2
14  14  13  12  11  10  9   8   7   6   5   4   3   2   1
\end{lstlisting}

(b) Based on the table above, the $(i \cdot j) \cong 1 (mod\ 15)$ pairs are:\\
$1 \cdot 1,\ 2 \cdot 8,\ 4 \cdot 4,\ 7 \cdot 13,\ 8 \cdot 2,\ 11 \cdot 11,\ 13 \cdot 7,\ 14 \cdot 14$
Then the multiplicative inverse (mod 15) of 1,$\dots$,14 are:

\begin{lstlisting}
Number   1   2   3   4   5   6   7   8   9   10  11  12  13  14
Inverse  1   8   X   4   X   X  13   2   X    X  11   X   7  14
\end{lstlisting}
%----------------------------------------------------------------------------------------

\end{document}