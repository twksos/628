\documentclass[11pt,twoside,a4paper]{article}
\usepackage{url}
\usepackage{amsmath,amsfonts,amsthm} % Math packages
\begin{document}
\title{A Technical Description and a Simplified Simulation of an Enigma Machine.}
\author{Guangcheng Wei}
\date{\normalsize\today}
\maketitle

\section{Introduction}
Information is an important issue in wars. To protect information, cryptograph is needed.
During the World War II, a new cryptograph algorithm has been invited to prevent known attacks on existing cryptograph back then. Frequency analysis was popular at that time. And Enigma Machine is proved resist frequency analysis.\cite{wiki:Enigma}
\section{Technical Description}
The Enigma Machine contains mainly 5 components:
Roters: 3 or more routers, each contains 26 positions.
Reflector: change singnal in pairs, make it able to use same process to encode and decode.
Plugboard: configurable plugboard
Keyboard: 26 key keyboard, stroke to input message.
Lamp panel: 26 lamps to display encoded/decoded message.

Encode process:
1. Inital Routers with ground setting.
2. Encode key with ground setting twice and retain key ciphertext.
3. Inital Routers with key.
4. Encode message to message ciphertext.
5. send key ciphertext and message ciphertext.

Decode process:
1. Inital Routers with ground setting.
2. Decode key ciphertext with ground setting twice and verify.
3. Inital Routers with key.
4. Decode message ciphertext to message.

\section{Simulation Design}
The simulation could be design in many ways. However, a pure software design would reflect the logic within. The simulation program would takes parameters mentioned before to initialize and encrypt plain text to cipher text.
Roters: $f(x) = x + r_i \mod 26$, for roter $i$ after $i$ times of use, $r_i$ will increase by 1.
Reflector: $f(x) = 26 - x$.
Plugboard: $f(x) = P[x]$.
Keyboard: input.
Lamp panel: output.

\section{Implementation}
\section{Result and Interpretation}
\section{Futher Thought}

\nocite{*}
\bibliographystyle{plain}
\bibliography{MID} 
\end{document}