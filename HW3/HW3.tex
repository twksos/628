%%%%%%%%%%%%%%%%%%%%%%%%%%%%%%%%%%%%%%%%%
% Short Sectioned Assignment
% LaTeX Template
% Version 1.0 (5/5/12)
%
% This template has been downloaded from:
% http://www.LaTeXTemplates.com
%
% Original author:
% Frits Wenneker (http://www.howtotex.com)
%
% License:
% CC BY-NC-SA 3.0 (http://creativecommons.org/licenses/by-nc-sa/3.0/)
%
% Author:
% Guangcheng Wei
%%%%%%%%%%%%%%%%%%%%%%%%%%%%%%%%%%%%%%%%%

%ref
% MI template http://www.cis.syr.edu/courses/cis428/DMI.pdf

% x mark from https://tex.stackexchange.com/questions/42619/x-mark-to-match-checkmark

% homework questions http://www.cis.syr.edu/courses/cis428/hw01.pdf
% Latex reference https://en.wikibooks.org/wiki/LaTeX
% slide http://www.cis.syr.edu/courses/cis428/slides/01mono.pdf

%----------------------------------------------------------------------------------------
%   PACKAGES AND OTHER DOCUMENT CONFIGURATIONS
%----------------------------------------------------------------------------------------

\documentclass[paper=a4, fontsize=11pt]{scrartcl} % A4 paper and 11pt font size

\usepackage[T1]{fontenc} % Use 8-bit encoding that has 256 glyphs
\usepackage{fourier} % Use the Adobe Utopia font for the document - comment this line to return to the LaTeX default
\usepackage[english]{babel} % English language/hyphenation
\usepackage[normalem]{ulem}
\usepackage{amssymb}% http://ctan.org/pkg/amssymb
\usepackage{pifont}% http://ctan.org/pkg/pifont
\newcommand{\xmark}{\ding{55}}%

\usepackage{amsmath,amsfonts,amsthm} % Math packages
\usepackage{listings} % Code package
\usepackage{lipsum} % Used for inserting dummy 'Lorem ipsum' text into the template

\usepackage{sectsty} % Allows customizing section commands
\allsectionsfont{\normalfont\scshape} % Make all sections centered, the default font and small caps

\usepackage{fancyhdr} % Custom headers and footers
\pagestyle{fancyplain} % Makes all pages in the document conform to the custom headers and footers
\fancyhead{} % No page header - if you want one, create it in the same way as the footers below
\fancyfoot[L]{} % Empty left footer
\fancyfoot[C]{} % Empty center footer
\fancyfoot[R]{\thepage} % Page numbering for right footer
\renewcommand{\headrulewidth}{0pt} % Remove header underlines
\renewcommand{\footrulewidth}{0pt} % Remove footer underlines
\setlength{\headheight}{13.6pt} % Customize the height of the header

\numberwithin{equation}{section} % Number equations within sections (i.e. 1.1, 1.2, 2.1, 2.2 instead of 1, 2, 3, 4)
\numberwithin{figure}{section} % Number figures within sections (i.e. 1.1, 1.2, 2.1, 2.2 instead of 1, 2, 3, 4)
% \numberwithin{table}{section} % Number tables within sections (i.e. 1.1, 1.2, 2.1, 2.2 instead of 1, 2, 3, 4)

\setlength\parindent{0pt} % Removes all indentation from paragraphs - comment this line for an assignment with lots of text

%----------------------------------------------------------------------------------------
%   TITLE SECTION
%----------------------------------------------------------------------------------------

\newcommand{\problem}[1]{\subsection *{Problem #1}}
\newcommand{\claim}{\textbf{Claim}: }
\newcommand{\basis}{\textbf{Basis:}: }
\newcommand{\inductiveStep}{\textbf{Inductive step:}: }
\newcommand{\pnl}{$ $\newline\\}
\newcommand{\N}{\mathbb{N}}
\newcommand{\Z}{\mathbb{Z}}
\newcommand{\horrule}[1]{\rule{\linewidth}{#1}} % Create horizontal rule command with 1 argument of height

\title{ 
\horrule{0.5pt} \\[0.4cm] % Thin top horizontal rule
\huge Homework 3: $\varphi$ and Friends \\ % The assignment title
\horrule{2pt} \\[0.5cm] % Thick bottom horizontal rule
}

\author{Guangcheng Wei (799233472)} % Your name
\date{\normalsize\today} % Today's date or a custom date

\begin{document}

\maketitle % Print the title

1.  Assignment: Homework 3\\

2a. Name \& email address (First author): Guangcheng Wei<gwei100@syr.edu> \\
2b. Name \& email address (Second author, if any):\\
\\
3a. Did you consult with anyone on parts of this assignment? (Yes/No): No\\
3b. If so, give the details of these consultations (e.g., who, what,    where)?\\
\\
4a. Did you consulted an outside source on parts of this assignment     (e.g., an Internet site)?  (Yes/No): Yes\\
4b. If so, give the details of these consultations (e.g., who, what,     where)?
\\
Assignment Latex Template(Frits Wenneker) http://www.latextemplates.com/template/short-sectioned-assignment\\
Latex Reference https://en.wikibooks.org/wiki/LaTeX\\
\\
Factor Calculator https://www.mathsisfun.com/numbers/factors-all-tool.html\#calc\\
\\
Tex file form Homework 1(Guangcheng Wei)\\
Tex file form Homework 2(Guangcheng Wei)\\
\\
Homework questions(Jim Royer) http://www.cis.syr.edu/courses/cis428/hw03.pdf\\
Slide(Jim Royer) http://www.cis.syr.edu/courses/cis428/slides/01mono.pdf\\
Slide(Jim Royer) http://www.cis.syr.edu/courses/cis428/slides/05rsa.pdf\\
\\
5.  The authors attest that the above is correct, (Yes/No): Yes\\

\pagebreak
\problem 1 
\renewcommand{\labelenumi}{(\alph{enumi})}
\begin{enumerate}
\item Suppose $p$ is a prime and $a, b \in \Z$ are such that $ab \cong 0\ (mod\ p)$ .\\
Show that $a \cong 0 (mod\ p)$ or $b \cong 0\ (mod\ p)$.\\
\begin{proof}
\pnl
$\because ab \cong 0\ (mod\ p)$\\
$\therefore p | (a\cdot b)$\\
$\therefore p | a$ or $p | b$\\
$\therefore a \cong 0 (mod\ p)$ or $b \cong 0\ (mod\ p)$\\\\
\end{proof}


\item Suppose $p$ is an odd prime. Show that the only two solutions of
$$x^2 \cong 1 (mod\ p)$$
are $$x \cong \pm 1 (mod\ p)$$
\begin{proof}
\pnl
$\because x^2 \cong 1 (mod\ p)$\\
$\therefore x^2-1 \cong 0 (mod\ p)$\\
$\therefore (x+1)(x-1) \cong 0 (mod\ p)$\\
By the proof in (a):\\
$(x+1) \cong 0 (mod\ p)$ or $(x-1) \cong 0\ (mod\ p)$\\
$\therefore x \cong 1 (mod\ p)$ or $x \cong -1 (mod\ p)$\\
$\therefore x \cong \pm 1 (mod\ p)$\\

\end{proof}
\end{enumerate}

\problem 2 Compute
\begin{enumerate}
\item $\varphi(41) = ?$\\
$\because$ 41 is a prime.\\
$\therefore \varphi(41) = 41 -1 = 40$\\
\item $\varphi(319) = ?$\\
$\because 319 = 11 * 29$, 11 and 29 are primes.\\
$\therefore \varphi(319) = (11-1) * (29-1) = 280$\\
\item $\varphi(360) = ?$\\
$\because 360 = 5 * 3^2 * 2^3$\\
$\therefore \varphi(360) = (5 - 1) * (9 - 3) * (8 - 4) = 96 $\\
\item $\varphi(1176) = ?$\\
$\because 1176 = 7^2 * 3 * 2^3$\\
$\therefore \varphi(1176) = (49 - 7) * (3 - 1) * (8 - 4) = 336 $\\
\end{enumerate}

\problem 3 Suppose $m > 0$ is odd. Prove that $\varphi(2m) = \varphi(m)$.
\begin{proof}
\pnl
Suppose m have prime factorization: $m = p_{1}^{k_1} \cdot p_{2}^{k_2} \cdot \ldots \cdot p_{n}^{k_n}$ where $p_{1} < p_{2} < \ldots < p_{n}$\\
$\because$ m is odd.\\
$\therefore \forall p \in \{p_{1}, p_{2}, \ldots, p_{n}\}$, $p \neq 2$\\
$\because$ Lemma 8.\\
$\therefore \varphi(m) = ( p_{1}^{k_1} -  p_{1}^{k_1 - 1}) \cdot ( p_{2}^{k_2} -  p_{2}^{k_2 - 1}) \cdot \ldots \cdot ( p_{n}^{k_n} -  p_{n}^{k_n - 1})$\\
and $\varphi(2m) = (2-1) \cdot ( p_{1}^{k_1} -  p_{1}^{k_1 - 1}) \cdot ( p_{2}^{k_2} -  p_{2}^{k_2 - 1}) \cdot \ldots \cdot ( p_{n}^{k_n} -  p_{n}^{k_n - 1})$\\
$\therefore \varphi(2m) =  (2 - 1) \cdot \varphi(m) = \varphi(m)$\\
\end{proof}

\problem 4 Suppose $m > 0$ is odd and
$$\Z^{*}_{m}=_{def} \left\{ a \in \{ 0, \dots , m - 1 \} : gcd(a, m) = 1 \right\}$$
\begin{enumerate}
\item Prove that if $a \in \Z^{*}_{m}$, then $((-a)\ mod\ m) \in \Z^{*}_{m}$.
\begin{proof}
\pnl
$\because a \in \Z^{*}_{m}$\\
$\therefore a < m$\\
$\therefore -a mod\ m = m-a$\\
To prove $((-a)\ mod\ m) \in \Z^{*}_{m}$, we need to show:\\
$gcd(m-a, m) = 1$ and $m-a \in \{ 1,\ldots,m-1 \}$\\
Which is $\exists x,y \in \Z \ (m-a)x+my = 1$ and $1 < m - a < m$ \\
$\because a < m$ and $a > 0$\\
$\therefore 0 < m - a < m $\\
$\therefore m-a \in \{ 1,\ldots,m-1 \}$\\
$\because gcd(a, m) = 1$\\
$\therefore \exists p,q \in \Z \ ap+mq = 1$\\
$\because (m-a)x+my = 1 \iff -xa + (x+y)m = 1$\\
$\therefore$ exist $x=-p$. $y = q -x = q+p$ for $(m-a)x+my = 1$\\
$\therefore \exists x,y \in \Z \ (m-a)x+my = 1$\\
$\therefore gcd(m-a, m) = 1$\\
$\therefore$ if $a \in \Z^{*}_{m}$, then $((-a)\ mod\ m) \in \Z^{*}_{m}$\\

\end{proof}
\item Suppose ${ a_1 <\cdots< a_{\varphi(m)} } = \Z^{*}_{m}$.\\
Prove that $(a_1 +\cdots+ a_{\varphi(m)}) \cong 0\ (mod\ m)$.
\begin{proof}
\pnl
CASE(I): $m = 1$ \\
$\therefore \Z^{*}_{m} = \phi$\\
$\therefore (a_1 +\cdots+ a_{\varphi(m)}) = 0$\\
$\therefore (a_1 +\cdots+ a_{\varphi(m)}) \cong 0\ (mod\ m)$\\

CASE(II): $m \neq 1$\\
$\because m$ is odd and $m > 0$\\
$\therefore m > 2$\\
$\therefore \varphi (m) is even$\\
Suppose $\varphi (m) = 2k$, $k \in \Z+$\\
$(a_1 +\cdots+ a_{\varphi(m)}) = (a_1 +\cdots+ a_{\varphi(2k)})$\\
$\because$ above proof in (a).\\
$\therefore a_1 + a_\varphi(2k) = a_2 + a_\varphi(2k-1) = \ldots = m$\\
$\therefore (a_1 +\cdots+ a_{\varphi(m)}) = km$\\
$\therefore (a_1 +\cdots+ a_{\varphi(m)}) \cong 0\ (mod\ m)$\\

For all cases, $(a_1 +\cdots+ a_{\varphi(m)}) \cong 0\ (mod\ m)$.\\
Therefore, $(a_1 +\cdots+ a_{\varphi(m)}) \cong 0\ (mod\ m)$.\\
\end{proof}
\end{enumerate}

\problem 5 
\begin{enumerate}
\item Suppose $p$ is a prime. Show that, for all $a \in Z, a^p \cong a\ (mod\ p)$.
\begin{proof}
\pnl
CASE(I): $p | a$, $a \in Z$\\
$\because\ p | a$\\
$\therefore\ a\ mod\ p = 0$ and $a^p\ mod\ p = 0$\\
$\therefore\ a^p \cong a\ (mod\ p)$\\
CASE(II): $p \nmid a$, $a \in Z$\\
With Fermat's Little Lemma:\\
$a^{p - 1} \equiv 1\ (mod\ p)$\\
Multiply $a$ on both side:\\
$\therefore\ a^p \equiv a\ (mod\ p)$\\
$\therefore\ a^p \cong a\ (mod\ p)$\\\\

For all cases, $a^p \cong a\ (mod\ p)$\\
Therefore, $\forall a \in Z, a^p \cong a\ (mod\ p)$

\end{proof}
\item Compute $2^{16049282}\ mod\ 131$\\
$\because 2^{130}\ mod\ 131 \equiv 1\ mod \ 131$\\
$2^{16049282}\ mod\ 131 = 2^{16049282\ mod\ 130}\ mod\ 131$\\
$= 2^2\ mod\ 131$\\
$= 4$\\
\end{enumerate}

%----------------------------------------------------------------------------------------

\end{document}