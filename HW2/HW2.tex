%%%%%%%%%%%%%%%%%%%%%%%%%%%%%%%%%%%%%%%%%
% Short Sectioned Assignment
% LaTeX Template
% Version 1.0 (5/5/12)
%
% This template has been downloaded from:
% http://www.LaTeXTemplates.com
%
% Original author:
% Frits Wenneker (http://www.howtotex.com)
%
% License:
% CC BY-NC-SA 3.0 (http://creativecommons.org/licenses/by-nc-sa/3.0/)
%
% Author:
% Guangcheng Wei
%%%%%%%%%%%%%%%%%%%%%%%%%%%%%%%%%%%%%%%%%

%ref
% MI template http://www.cis.syr.edu/courses/cis428/DMI.pdf

% x mark from https://tex.stackexchange.com/questions/42619/x-mark-to-match-checkmark

% homework questions http://www.cis.syr.edu/courses/cis428/hw01.pdf
% Latex reference https://en.wikibooks.org/wiki/LaTeX
% slide http://www.cis.syr.edu/courses/cis428/slides/01mono.pdf

%----------------------------------------------------------------------------------------
%   PACKAGES AND OTHER DOCUMENT CONFIGURATIONS
%----------------------------------------------------------------------------------------

\documentclass[paper=a4, fontsize=11pt]{scrartcl} % A4 paper and 11pt font size

\usepackage[T1]{fontenc} % Use 8-bit encoding that has 256 glyphs
\usepackage{fourier} % Use the Adobe Utopia font for the document - comment this line to return to the LaTeX default
\usepackage[english]{babel} % English language/hyphenation

\usepackage{amssymb}% http://ctan.org/pkg/amssymb
\usepackage{pifont}% http://ctan.org/pkg/pifont
\newcommand{\xmark}{\ding{55}}%

\usepackage{amsmath,amsfonts,amsthm} % Math packages
\usepackage{listings} % Code package
\usepackage{lipsum} % Used for inserting dummy 'Lorem ipsum' text into the template

\usepackage{sectsty} % Allows customizing section commands
\allsectionsfont{\normalfont\scshape} % Make all sections centered, the default font and small caps

\usepackage{fancyhdr} % Custom headers and footers
\pagestyle{fancyplain} % Makes all pages in the document conform to the custom headers and footers
\fancyhead{} % No page header - if you want one, create it in the same way as the footers below
\fancyfoot[L]{} % Empty left footer
\fancyfoot[C]{} % Empty center footer
\fancyfoot[R]{\thepage} % Page numbering for right footer
\renewcommand{\headrulewidth}{0pt} % Remove header underlines
\renewcommand{\footrulewidth}{0pt} % Remove footer underlines
\setlength{\headheight}{13.6pt} % Customize the height of the header

\numberwithin{equation}{section} % Number equations within sections (i.e. 1.1, 1.2, 2.1, 2.2 instead of 1, 2, 3, 4)
\numberwithin{figure}{section} % Number figures within sections (i.e. 1.1, 1.2, 2.1, 2.2 instead of 1, 2, 3, 4)
% \numberwithin{table}{section} % Number tables within sections (i.e. 1.1, 1.2, 2.1, 2.2 instead of 1, 2, 3, 4)

\setlength\parindent{0pt} % Removes all indentation from paragraphs - comment this line for an assignment with lots of text

%----------------------------------------------------------------------------------------
%   TITLE SECTION
%----------------------------------------------------------------------------------------

\newcommand{\problem}[1]{\subsection *{Problem #1}}
\newcommand{\claim}{\textbf{Claim}: }
\newcommand{\basis}{\textbf{Basis:}: }
\newcommand{\inductiveStep}{\textbf{Inductive step:}: }
\newcommand{\pnl}{$ $\newline\\}
\newcommand{\N}{\mathbb{N}}
\newcommand{\Z}{\mathbb{Z}}
\newcommand{\horrule}[1]{\rule{\linewidth}{#1}} % Create horizontal rule command with 1 argument of height

\title{ 
\horrule{0.5pt} \\[0.4cm] % Thin top horizontal rule
\huge Homework 1: A Math Warm-Up \\ % The assignment title
\horrule{2pt} \\[0.5cm] % Thick bottom horizontal rule
}

\author{Guangcheng Wei (799233472)} % Your name
\date{\normalsize\today} % Today's date or a custom date

\begin{document}

\maketitle % Print the title

1.  Assignment: Homework 1\\

2a. Name \& email address (First author): Guangcheng Wei<gwei100@syr.edu> \\
2b. Name \& email address (Second author, if any):\\
\\
3a. Did you consult with anyone on parts of this assignment? (Yes/No): No\\
3b. If so, give the details of these consultations (e.g., who, what,    where)?\\
\\
4a. Did you consulted an outside source on parts of this assignment     (e.g., an Internet site)?  (Yes/No): Yes
4b. If so, give the details of these consultations (e.g., who, what,     where)?
\\
Assignment Latex Template(Frits Wenneker) http://www.latextemplates.com/template/short-sectioned-assignment\\
Latex Reference https://en.wikibooks.org/wiki/LaTeX\\
\\
Homework 1(Guangcheng Wei)\\
\\
Homework questions(Jim Royer) http://www.cis.syr.edu/courses/cis428/hw02.pdf\\
Slide(Jim Royer) http://www.cis.syr.edu/courses/cis428/slides/01mono.pdf\\
Slide(Jim Royer) http://www.cis.syr.edu/courses/cis428/slides/02poly.pdf\\
Char frequency http://www.data-compression.com/english.html\\
\\
5.  The authors attest that the above is correct, (Yes/No): Yes\\

\pagebreak
\problem 1 The following is the ciphertext of a message encrypted with a shift
cipher.
\begin{lstlisting}
XLCJS LOLWT EEWPV PJDSP VPAET ETYPD NCZHL YOPGP
CJEST YRESL EXLCJ DLTOE SPQPO DHPCP DFCPE ZVYZH
\end{lstlisting}
Your job: (a) Find the key and decrypt the message, and (b) describe
the steps you went through in finding the key.\\\\

(a) The key is 11 and the decrypted message is: \\
\begin{lstlisting}
MARYH ADALI TTLEK EYSHE KEPTI TINES CROWA NDEVE
RYTHI NGTHA TMARY SAIDT HEFED SWERE SURET OKNOW
\end{lstlisting}
The message should be:\\
MARY HAD A LITTLE KEY. SHE KEPT IT IN ESCROW AND EVERYTHING THAT MARY SAID THE FEDS WERE SURE TO KNOW.\\
\\
(b) The step I used are:\\
1. implement encrypt and decrypt of shift cipher\\
\begin{lstlisting}
    //encrypt: mod(p + shift, 26) = c = mod(c, 26)
    //decrypt: mod(c - shift, 26) = p = mod(p, 26)
\end{lstlisting}
2. Write the logic of solve shift cipher:\\
\begin{lstlisting}
    //solve:   mod(shift, 26) = mod(c - p, 26)
\end{lstlisting}
Then the solve process is easy when given plaintext and ciphertext.\\
\begin{lstlisting}
function solveShift(cstr, p, c) {
    //encrypt: mod(p + shift, 26) = c = mod(c, 26)
    //decrypt: mod(c - shift, 26) = p = mod(p, 26)
    
    //solve:   mod(shift, 26) = mod(c - p, 26)
    const shift = mod(c.charCodeAt(0) - p.charCodeAt(0), 26);
    const pstr = decShift(cstr, shift);
    return {key: shift, plainText: pstr};
}
\end{lstlisting}

3. Frequency analysis of cipher text.\\
\begin{lstlisting}
    const problem1cipherText = 'XLCJS LOLWT EEWPV PJDSP VPAET ETYPD NCZHL ' +
        'YOPGP CJEST YRESL EXLCJ DLTOE SPQPO DHPCP DFCPE ZVYZH';
    const problem1FreqCount = freqCount(problem1cipherText);
    console.log(problem1FreqCount);

Result:     
[ { key: 'F', count: 1 },

  ...

  { key: 'C', count: 6 },
  { key: 'L', count: 7 },
  { key: 'E', count: 9 },
  { key: 'P', count: 12 } ]
\end{lstlisting}
4. Try let the highest frequency char P in cipher text to be from E, to see if the decoded output is meaningful.\\
\begin{lstlisting}
const problem1Result = solveShift(problem1cipherText, 'E', 'P');
console.log(problem1Result);
\end{lstlisting}
5. The out put is meaningful, accept the key = 11 and test encrypt plain text with 11. The result of encrypt is same as the cipher text.\\
6. Split the words in plain text use knowledge in english.\\


\problem 2 The following is the ciphertext of a message encrypted with an affine
cipher.
\begin{lstlisting}
FPFKR NEHAF LJHZH KRPHY EVURH HDVSF MHAFL JYRVY MR
\end{lstlisting}
Your job: (a) Find the key and decrypt the message, and (b) describe
the steps you went through in finding the key.\\\\
(a) The key is $a=3, b=5$ in $c=a*p+b$. and the decrypted message is:
\begin{lstlisting}
AMATE URSHA CKSYS TEMSP ROFES SIONA LSHAC KPEOP LE
\end{lstlisting}
The message should be:\\
AMATEUR SHACK SYSTEMS PROFESSIONAL SHACK PEOPLE\\
\\
(b) The step I used are:
1. implement encrypt and decrypt of affine cipher\\
\begin{lstlisting}
    //encrypt: mod(a * p + b, 26) = c = mod(c, 26)
    //decrypt: mod((c - shift) * INV[a], 26) = p = mod(p, 26)
\end{lstlisting}
2. Clearify the logic of solve affine cipher:\\
As the choice of $a$ is limited to $a \in \{ x| mod(x * y) = 1, y \in \Z \ and \ 0 \leq y \leq 25\}$\\
When we have 2 pair of plain text to cipher text $c_0 = enc_{affine}(p_0), c_1 = enc_{affine}(p_1)$, we can have:
$$ c_0 = c_0\ mod\ 26 = (ap_0+b)\ mod\ 26$$ 
$$ c_1 = c_1\ mod\ 26 = (ap_1+b)\ mod\ 26$$
by subtract one from another, we can elimiate variable $b$:\\
$$(c_0 - c_1)\ mod\ 26 = (a(p_0-p_1))\ mod\ 26$$
$$(c_0 - c_1)\ mod\ 26 = (a(p_0-p_1))\ mod\ 26$$
Because choice of a is limited, we can try those choices and get $a$s or discard this pair when no $a$ matches.\\
Then for each $a$, we can calculate $b$ from both $b_0= (c_0 - a*p_0)\ mod \ 26$ and $b_1= (c_1 - a*p_1)\ mod \ 26$\\
If $b_0 = b_1$ we accept it as $b$. Otherwise, discard this $a$.\\\\
After go through the process above, we will have a set of variables. Then we try each of those possible variables and see if the outcome is meaningful.\\

3. Frequency analysis of cipher text.\\
\begin{lstlisting}
[ { key: 'S', count: 1 },

  ...

  { key: 'V', count: 3 },
  { key: 'Y', count: 3 },
  { key: 'R', count: 5 },
  { key: 'F', count: 5 },
  { key: 'H', count: 7 } ]
\end{lstlisting}
4. Try let the highest frequency char pair H, F in cipher text to be from ['E', 'T', 'A', 'O', 'N', 'I', 'S']. Run the solve function for each permutation and collect the result.\\
\begin{lstlisting}
    const highFreq = ['E', 'T', 'A', 'O', 'N', 'I', 'S'];
    let results = [];
    for (let i = 0; i < 6; i++) {
        for (let j = 0; j < 6; j++) {
            if (j == i) continue;
            const firstP = highFreq[i];
            const secondP = highFreq[j];
            const firstC = 'H';
            const secondC = 'F';
            const problem2Result = solveAffine(
                problem2cipherText,
                [firstP, secondP],
                [firstC, secondC]
            );
            if (problem2Result) results = results.concat(problem2Result);
        }
    }
    console.log(results);
\end{lstlisting}
5. There is no really meaningful result in the output. One interesting output is:\\
\begin{lstlisting}
[
  ...

  { pv: [ 'O', 'I' ],
    cv: [ 'H', 'F' ],
    variable: { a: '9', b: 11 },
    plainText: 'IMIXSGFOTIAUOQOXSMONFEBSOOCEVIDOTIAUNSENDS' },
  
  ...
]
\end{lstlisting}
It contains 'I mix', 'Mon', 'Feb', 'I do', 'Ends'. However, it is not meaningful.\\

6. Try let second highest frequency char pair F, R in cipher text to be from ['E', 'T', 'A', 'O', 'N', 'I', 'S'].\\
\begin{lstlisting}
    ...
            const firstC = 'R';
            const secondC = 'F';
    ...
\end{lstlisting}
7. The first output is meaningful.\\
\begin{lstlisting}
[ { pv: [ 'E', 'A' ],
    cv: [ 'R', 'F' ],
    variable: { a: '3', b: 5 },
    plainText: 'AMATEURSHACKSYSTEMSPROFESSIONALSHACKPEOPLE' },

    ...
]
\end{lstlisting}
9. Accept the key = { a: 3, b: 5 } and test encrypt plain text with $a=3, b=5$. The result of encrypt is same as the cipher text.\\
10. Split the words in plain text use knowledge in english.\\



\problem 3 Suppose we have a password policy that passwords must:
\begin{enumerate}
\item consist of upper and lower case letters and digits; and
\item be between 6 and 8 characters long.
\end{enumerate}

\renewcommand{\labelenumi}{(\alph{enumi})}
\begin{enumerate}
\item How many passwords are there under these rules?
\item How many passwords are there if we add the rule that no letter may be used twice.
\end{enumerate}
(a)
$N_{password} = N_{password\ length\ 6} + N_{password\ length\ 7} + N_{password\ length\ 8}$\\
$= (26 + 26 + 10)^{6} +(26 + 26 + 10)^{7}+(26 + 26 + 10)^{8}$\\
$= (26 + 26 + 10)^{6} +(26 + 26 + 10)^{7}+(26 + 26 + 10)^{8}$\\
$= 62^6 + 62^7 + 62^8$\\
$= 221918520426688$\\

(b)
$N_{password} = N_{password\ length\ 6} + N_{password\ length\ 7} + N_{password\ length\ 8}$\\
$= P(62, 6) + P(62, 7) + P(62, 8)$\\
$= 62 * 61 * 60 * 59 * 58 * 57 +  62 * 61 * 60 * 59 * 58 * 57 * 56 +  62 * 61 * 60 * 59 * 58 * 57 * 56 * 55$\\
$= 138848807594160$\\

\problem 4 A pre-season poll will rank the top 25 teams from the 830 soccer
teams in the country. How many different rankings are possible?\\\\
$Ranking Counts = P(830, 25)$\\
$= \frac{830!}{(830-25)!}$\\
$= 830 * 829 * \cdots * 806$\\
$= 6.5828003 * 10^{72}$\\

\problem 5 We have a box containing: 4 green balls, 6 blue balls, 3 yellow balls,
and 7 red balls. A ball is selected at random from the box. Compute
the probability that the selected ball is:\\\\

\renewcommand{\labelenumi}{(\alph{enumi})}
\begin{enumerate}
\item blue
\item yellow
\item not red
\item yellow or blue
\item not green and not red
\item either green, blue, or yellow
\end{enumerate}

$P(blue) = \frac{6}{20} = 0.3$\\
$P(green) = \frac{4}{20} = 0.2$\\
$P(yellow) = \frac{3}{20} = 0.15$\\
$P(red) = \frac{7}{20} = 0.35$\\

\begin{enumerate}
\item $P(blue) = 0.3$
\item $P(yellow) = 0.2$
\item $P(not\ red) = 1 - P(red) = 1 - 0.35 = 0.65$
\item $P(yellow\ or\ blue) = P(yellow) + P(blue) = 0.15 + 0.3 = 0.45$
\item $P(not\ green\ and\ not\ red) = 1 - P(green) - P(red) = 1 - 0.2 - 0.35 = 0.45$
\item $P(either\ green,\ blue,\ or\ yellow) = P(green) + P(blue) + P(yellow) = 0.2 + 0.3 + 0.15 = 0.65$
\end{enumerate}

\problem 6 In a survey of 1000 people, categorized as smokers ($S$) or nonsmokers
($\neg S$) and as with respiratory problems ($R$) or without respiratory problems
($\neg R$). The results are given in Table 1.
Based on these results, determine whether the follow pairs of events are independent:

\begin{table}[h]
  \centering
  \begin{tabular}{c | c c}
    & $R$ & $\neg R$ \\ \hline
    $S$ & 520 & 180 \\
    $\neg S$ & 77 & 223 \\
  \end{tabular}
  \caption{Survey Results}

\end{table}

\begin{enumerate}
\item $S$ and $\neg S$
\item $R$ and $\neg R$
\item $S$ and $R$
\item $S$ and $\neg R$
\item $\neg S$ and $R$
\item $\neg S$ and $\neg R$
\end{enumerate}


$P(S \bigcap R) = \frac{520}{1000} = 0.52$\\\\
$P(S \bigcap \neg R) = \frac{180}{1000} = 0.18$\\\\
$P(\neg S \bigcap R) = \frac{77}{1000} = 0.077$\\\\
$P(\neg S \bigcap \neg R) = \frac{223}{1000} = 0.223$\\\\
$P(S) = \frac{520 + 180}{1000} = 0.7$\\\\
$P(\neg S) = \frac{77 + 223}{1000} = 0.3$\\\\
$P(R) = \frac{520 + 77}{1000} = 0.597$\\\\
$P(\neg R) = \frac{180 + 223}{1000} = 0.403$\\\\

\begin{enumerate}
\item $S$ and $\neg S$ are not independent.\\
$P(S \bigcap \neg S) = 0\\ P(S) \cdot P(\neg S) = 0.7 \cdot 0.3= 0.21$\\
$P(S \bigcap \neg S) \neq P(S) \cdot P(\neg S)$\\

\item $R$ and $\neg R$ are not independent.\\
$P(R \bigcap \neg R) = 0\\ P(R) \cdot P(\neg R) = 0.597 \cdot 0.403 = 0.240591$\\
$P(R \bigcap \neg R) \neq P(R) \cdot P(\neg R)$\\
\item $S$ and $R$ are not independent.\\
$P(S \bigcap R) = 0.52\\ P(S) \cdot P(R) = 0.7 \cdot 0.597 = 0.4179$\\
$P(S \bigcap R) \neq P(S) \cdot P(R)$\\
\item $S$ and $\neg R$ are not independent.\\
$P(S \bigcap \neg R) = 0.18\\ P(S) \cdot P(\neg R) = 0.7 \cdot 0.403 = 0.2821$\\
$P(S \bigcap \neg R) \neq P(S) \cdot P(\neg R)$\\
\item $\neg S$ and $R$ are not independent.\\
$P(\neg S \bigcap R) = 0.077\\ P(\neg S) \cdot P(R) = 0.3 \cdot 0.597 = 0.1791$\\
$P(\neg S \bigcap R) \neq P(\neg S) \cdot P(R)$\\
\item $\neg S$ and $\neg R$ are not independent.\\
$P(\neg S \bigcap \neg R) = 0.223\\ P(\neg S) \cdot P(\neg R) = 0.3 \cdot 0.403= 0.1209$\\
$P(\neg S \bigcap \neg R) \neq P(\neg S) \cdot P(\neg R)$\\
\end{enumerate}

%----------------------------------------------------------------------------------------

\end{document}